\documentclass[compacto]{aleph-formato}

\usepackage{aleph-comandos-beta}

\autor{Daniel Lara}
\materia{Analisis Funcional}
\tema{Tarea 1}
\fecha{Diciembre, 2023}
\codigomateria{M101}
\subtema{ }

\fuente{mathpazo}

\begin{document}

% \vspace*{-0.1mm}
\encabezado

\begin{teo}
    En un espacio métrico de dimensión finita todas las normas son equivalentes.
\end{teo}

\begin{obs}
    En el caso de dimensión infinita, no todas las normas son equivalentes. Por ejemplo, en el espacio $C([0,1])$ con la norma del supremo, la norma $\norma{}_1$ no es equivalente a la norma del supremo.
\end{obs}

\begin{ejer}
    Sean $(X,d_1)$ y $(X,d_2)$ dos espacios métricos y sean $a$ y $b$ números positivos tales que para todo $x,y\in X$, se cumple que
    \[
        a d_1(x,y) \leq d_2(x,y) \leq b d_1(x,y).
    \]
    Muestre que si $\suc{x_n}$ es una sucesión de Cauchy en $(X,d_1)$ entonces también es una sucesión de Cauchy en $(X,d_2)$.
\end{ejer}

\begin{esquema}
Gracias a la desigualdad de la hipótesis se puede acotar la sucesión en la métrica $d_2$ con la métrica $d_1$, luego el resultado es evidente. 
\end{esquema}

\begin{ejer}
    Muestre que el subespacio $Y\subseteq C([a,b])$ que consiste de todas las funciones $u\in C([a,b])$ tales que $u(a)=u(b)$, es completo. 
\end{ejer}

\begin{esquema}
    Mostremos que $Y$ es cerrado en $(C([a,b]),d_\infty)$, por lo tanto, completo. En efecto, si $\suc{f_n}$ está en $Y$ y como la convergencia del espacio es uniforme, entonces $f(a)=f(b)$. Por lo tanto, el espacio es cerrado en un espacio completo, lo que implica que $Y$ es completo con la métrica $d_\infty$
\end{esquema}

\end{document}



