\documentclass[11pt,a4paper]{ltxdoc} 
\usepackage[spanish,es-noindentfirst,es-tabla]{babel}

\usepackage[utf8]{inputenc}
\usepackage[T1]{fontenc}
\usepackage{graphicx}
\usepackage{xcolor}
\usepackage{float}
\usepackage[margin=2.5cm,left=3.5cm]{geometry}
\usepackage{mathpazo}
\usepackage{changelog}

\setlength{\parskip}{0.2\baselineskip}
\renewcommand{\baselinestretch}{1.1}

\newcommand{\file}[1]{\texttt{#1}}
\newcommand{\option}[1]{\texttt{#1}}
\newcommand{\package}[1]{\texttt{#1}}

\title{\file{aleph-notas.cls}}
\author{Proyecto Alephsub0\\ Daniel Lara\footnote{Facultad de Ciencias, Escuela Politécnica Nacional}}
\date{2023-12-26\\ Versión 2.0}

\usepackage[colorlinks,linkcolor=teal,urlcolor=teal,
   citecolor=black,bookmarks=true]{hyperref}
\usepackage{url}

\begin{document}
 
\maketitle

\begin{abstract}
    \file{aleph-formato.cls} es una clase creada para dar formato a tareas realizadas en \LaTeX. Esta clase fue generada dentro del proyecto Alephsub0 (\url{https://www.alephsub0.org/}).
\end{abstract}

\tableofcontents

\section{Introducción}

La clase \file{aleph-foramto.cls} es está inspirada \texttt{aleph-notas.cls} en parte del conjunto de clases y paquetes creados por Andrés Merino dentro de su proyecto personal Alephsub0.

\section{Uso}

Para cargar la clase se utiliza: \cs{documentclass}\oarg{opciones}|{aleph-formato}| con las opciones acordes al formato que se desee.


\subsection{Opciones}

Las opciones de la clase son las siguientes:
\begin{description}
    \item[|9pt|,|10pt|, |11pt|, |12pt|] ajustan el tamaño de fuente. Por defecto, se usa |10pt|.
    \item[|amplio|, |compacto|, |A4|, |A4-s|, |A5|] genera la geometría de las notas predeterminada, es decir, tamaño de página y márgenes. Las dimensiones generadas por por estas opciones están dadas en la Tabla~\ref{tab:01}. Por defecto, se usa |compacto|.
    \item[|comentarios|, |codigo|] muestran el contenido de los ambientes opcionales |comentario| y |código|. Por defecto, no se muestra el contenido de estos ambientes.
\end{description}
  
\begin{table}[ht]
    \centering
    \begin{tabular}{cccccc}\hline
        Opción & Dimensiones & Interno & Externo & Superior & Inferior \\\hline
        |amplio| & 195mm$\times$265mm & 2.2cm & 2.2cm & 2.25cm & 2.25cm\\
        |compacto| & 160mm$\times$240mm & 1.7cm & 1.7cm & 2cm & 2cm\\
        |A4| & A4 & 2.2cm & 2.2cm & 2.25cm & 2.25cm\\
        |A4-s| & A4 & 1cm & 1cm & 1cm & 1.8cm\\
        |A5| & A5 & 2.2cm & 1cm & 2.25cm & 2.25cm\\
        \hline
    \end{tabular}
    \caption{Geometría de página predefinida.}
    \label{tab:01}
\end{table}

\subsection{Colores}

Las clase trabaja con un color básico:
\begin{description}
    \item[|colordef|] es el color preestablecido para los ambientes de teoremas y notas al margen. El color predefinido por la clase es negro.

\end{description}
Se puede cambiar fácilmente estos colores con el comando\\
    \hspace*{3em}\cs{definecolor}|{colordef}|\marg{formato de color}\marg{color}\\
    \hspace*{3em}\cs{definecolor}|{colortext}|\marg{formato de color}\marg{color}

\section{Comando para tipografía}

Para esta nueva versión se incluye el comando \verb|\fuente|

\DescribeMacro{\fuente} 
    El comando fuente tiene el formato\\
        \hspace*{3em}\cs{fuente}\marg{nombre fuente},\\
    el \meta{nnombre fuente} especifica el tipo de paquete de fuente incluido en el documento, las opciones son: mathpazo y monstserrat.

\subsection{Comandos de datos informativos para las notas}

\DescribeMacro{\institucion} 
    El comando institucion tiene el formato\\
        \hspace*{3em}\cs{institucion}\marg{nombre de la institución}.\\
    este comando es opcional.

\DescribeMacro{\autor} 
    El comando autor tiene el formato\\
        \hspace*{3em}\cs{autor}\marg{nombre autor}.\\
        este comando es obligatorio.

\DescribeMacro{\materia}
\DescribeMacro{\codigomateria}
\DescribeMacro{\semetre}
\DescribeMacro{\fecha} 
\DescribeMacro{\tema}
\DescribeMacro{\subtema}
    
\subsection{Comandos varios}

\DescribeMacro{\encabezado}
    El comando \cmd{\encabezado} genera el encabezado de la página, no tiene ninguna opción.

\newpage
\subsection{Estilo de teoremas}

\DescribeEnv{ejem}
\DescribeEnv{obs}
\DescribeEnv{prop}
\DescribeEnv{cor}
\DescribeEnv{lem}
\DescribeEnv{teo}
\DescribeEnv{defi}
\DescribeEnv{axioma}
\DescribeEnv{ejer}
    Existen dos estilos de teoremas definidos: recuadro sin título aparte y recuadro con título aparte. Los ambientes predefinidos son: 
\begin{description}
    \item[|ejem|] para ejemplos, no utiliza recuadro, se numeran según la sección.
    \item[|obs|] para observaciones, no utiliza recuadro, por defecto no se numera a menos que se tenga la opción |numobs|.
    \item[|prop|{,} |cor|{,} |lem|] para proposiciones, corolarios y lemas, utiliza recuadro sin título aparte. Se numeran según la sección.
    \item[|teo|] para teoremas, utiliza recuadro con título aparte izquierdo. Se numeran continuando |prop|.
    \item[|defi|] para definiciones, utiliza recuadro con título aparte izquierdo. Se numeran según la sección.
    \item[|axioma|] para axiomas, utiliza recuadro con título aparte izquierdo. Se numeran según el sección.
    \item[|ejer|] para ejercicios, utiliza recuadro sin título aparte con la opción. Se numeran según el sección.
\end{description}

\subsection{Estilo de advertencia}

\DescribeEnv{advertencia}
    Este ambiente genera un recuadro amarillo sin título, pensado originalmente para colocar advertencias llamativas. Se pueden generar ambientes similares a este bajo la siguiente sintaxis:\\
    \hspace*{3em}\cs{newtcolorbox}\marg{nombre del ambiente}\verb"{postit}"
    
También se pueden generar más ambientes de teoremas, con otros formatos y colores, siguiendo los ejemplos que se muestran en el paquete \package{aleph-libro.cls} (\url{https://www.alephsub0.org/recursos/}).

\subsection{Problemas}

\begin{itemize}
\item 
    La versión actual trabaja bien en la versión de TeXLive 2019 en adelante (específicamente, con el paquete \package{tcolorcox} v4.20). Si se usa una versión anterior, existe una incompatibilidad con la actualización del paquete. Para utilizar versiones anteriores de ese paquete, es necesario cambiar:
    \begin{itemize}
        \item |tcbcolback| por |tcbcol@back|
        \item |tcbcolframe| por |tcbcol@frame|
    \end{itemize}
\end{itemize}

Cualquier problema adicional, por favor reportarlo a\\ 
\url{mat.andresmerino@gmail.com}.

\begin{changelog}[author=Daniel Lara,
    sectioncmd=\section]
    % version 2.0
    \begin{version}[v=2.0,
        date=2023-11-27]
        \removed
        \item Se elimina el formato clásico para el estilo de los teoremas.
        \added 
        \item Se añade la posibilidad de seleccionar fuentes: Palatino Linotype (mathpaso) y Monstserrat
        \changed 
        \item Se modifica el diseño del encabezado.
    \end{version}
    % version 1.1
    \begin{version}[v=1.1,
    date=2021]
        \added
        \item Se incluye estilos de formato para ambientes: fclasico, fnuevo
    \end{version}
    % version 1.0
    \shortversion{v=1.0,
    date=2019-05-11,
    changes=Primera versión del paquete \texttt{formato-deberes}}
\end{changelog}

\newpage
\DocInput{aleph-formato.dtx}

\end{document}
